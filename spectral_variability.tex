\documentclass[12pt, letterpaper]{article}
\usepackage{amsmath}


\newcommand{\acronym}[1]{{\small{#1}}}
\newcommand{\project}[1]{\textsl{#1}}
\newcommand{\Avast}{\project{A$\!$vast}}

\newcommand{\unit}[1]{\mathrm{#1}}
\newcommand{\m}{\unit{m}}
\newcommand{\s}{\unit{s}}
\newcommand{\mps}{\m\,\s^{-1}}

\newcommand{\dd}{\mathrm{d}}
\newcommand{\given}{\,|\,}
\newcommand{\normal}{{\mathcal{N}}}
\newcommand{\com}{\mathrm{cm}}
\newcommand{\surf}{\mathrm{sf}}

\setlength{\parindent}{1.0\baselineskip}
\linespread{1.09}
\raggedbottom
\sloppy
\sloppypar
\frenchspacing

\begin{document}

\section*{Observing time variability of stellar spectra with \Avast}

\noindent
by DWH, MB, others

\paragraph{Abstract:}


\section{Introduction}


\section{Method}

The most important assumptions we make in this work are the following:
\begin{itemize}\itemsep=0ex
\item There are $N$ multiple extracted spectra of the same star, taken
  at (fictitious) Solar-System barycentric times (epochs) $t_n$. Each
  of these spectra has been continuum normalized properly, and is
  represented on a grid of $M$ wavelength pixels $\lambda_m$ in the
  (fictitious) Solar-System barycentric rest frame. We assume (perhaps
  naively) that both the times $t_n$ and wavelengths $\lambda_m$ are
  very precisely known.
\item At each pixel $m$ of each spectrum $n$ there is Gaussian noise,
  drawn from a distribution with zero mean and known variance
  $\sigma^2_{ij}$.
\item At each epoch $n$, there is a radial velocity displacement
  $\Delta v_{n}$ of the star. This shift may be a combination
  of ``true'' center-of-mass radial velocity induced by stellar companions
  and/or (integrated) velocity fields in the stellar photosphere. Additionally,
  there is also some kind of spectral distortion.  That is, there is a
  composite Doppler shift and a spectral change at every epoch.
\item We expect some aspects of the spectral distortions to be sparse
  in wavelength-space.  That is, we expect only a few, isolated lines
  to participate in (say) a distortion arising from chromospheric stellar activity.
\item We expect other aspects of the spectral distortions to be low
  rank; these might be caused by systematic problems in the
  spectrograph, or in the continuum normalization procedures.
\item We expect that the center-of-mass velocity is set by some kind
  of kinematic or dynamical model (such as the two-body orbit problem)
  and any distortions to be set by some kind of stochastic process.
\end{itemize}

Our model (likelihood) is described by
\begin{eqnarray}
  %y_{ij} &=& s_j f(\lambda'_{ij}) + \mathrm{systematics}_{ij} + \mathrm{noise}_{ij}
   y_{ij} &=& s_j f(\lambda'_{ij})
  \\
  \lambda'_{ij} &=& \lambda_i\,\sqrt{\frac{1 - \beta_j}{1 + \beta_j}}
  \\
  \beta &\equiv& \frac{\Delta v}{c}
  \quad , 
\end{eqnarray}
where the $y_{ij}$ is the continuum-normalized flux in pixel $i$ of
spectrum $j$, $s_j$ is a constant scaling factor related to the observed
flux level at epoch $j$, $f()$ is a function of
wavelength describing the intrinsic stellar spectrum, and
$\lambda'_{ij}$ is the stellar rest-frame wavelength
corresponding to the barycentric wavelength $\lambda_i$ observed at
(dimensionless) velocity $\beta_j$.
%There are both (independent-ish, pixel) noise contributions and
%(covariant, low-rank) systematics contributions to the observed
%signal.
We represent or generate these functions and latent variables
with the following:
\begin{eqnarray}
  f(\lambda) &=& \sum_{m=0}^{M-1} a_m\,g_m(\lambda)
  \\
  g_m(\lambda) &=& \left\{
        \begin{array}{ll}
            0 & \quad \lambda \leq x_m -\Delta x \\
            1 - \frac{\lambda - x_m}{\Delta x} & \quad x_m - \Delta x \leq \lambda \leq x_m \\
            1 - \frac{x_m - \lambda}{\Delta x} & \quad x_m \leq \lambda \leq x_m + \Delta x \\
            0 & \quad \lambda \geq x_m + \Delta x
        \end{array}
    \right.
  \quad ,
\end{eqnarray}
where the $a_m$ are expansion coefficients and the $g_m(\lambda)$
are basis functions corresponding to a triangle centered at $x_m$ with half-width 
$\Delta x$. For this implementation, we chose $x_m$ spaced evenly in $\log \lambda$, 
with $\Delta x$ equal to the spacing between each $x_m$.
%We also have both systematics and noise, which we model with covariant
%and independent Gaussians:
%\begin{eqnarray}
%  \mathrm{systematics}_{ij} &=& \sum_{t=0}^{T-1} A_{jt}\,G_t(\lambda'_{ij})
%  \\
%  p(A_{jt}\given V_t) &=& \normal(A_{jt}\given 0,V_t)
%  \\
%  p(\mathrm{noise}_{ij}) &=& \normal(\mathrm{noise}_{ij}\given 0,\sigma_{ij}^2)
%  \quad ,
%\end{eqnarray}
%where the $A_{jt}$ are systematic noise amplitudes drawn from a
%Gaussians of variance $V_t$, the $G_{t}()$ are systematics functions
%(eigenfunctions, perhaps), and the pixel noise is all drawn
%independently (but heteroskedastically).
Note that nothing about this model requires that the wavelength grid
$\lambda_i$ be the same from spectrum to spectrum (that is, the
wavelength grid can depend on $j$), and there is no requirement that
the data at different epochs be similar in signal-to-noise. We do,
effectively, at this point, assume that instrument resolution is
identical (or very similar) from epoch to epoch.

The free parameters to be optimized are the scaling factors $s_j$, the basis expansion
coefficients $a_m$, and the radial velocities $\beta_j$. We use the derivative of 
$y$ to perform this optimization:
\begin{eqnarray}
  %y_{ij} &=& s_j f(\lambda'_{ij}) + \mathrm{systematics}_{ij} + \mathrm{noise}_{ij}
   \frac{dy_{ij}}{d \mathrm{par}} &=& \Big((\frac{\partial y_{ij}}{\partial s_j})^2 + 
   \sum_{m=0}^{M-1}(\frac{\partial y_{ij}}{\partial f} \frac{\partial f}{\partial a_m})^2 + 
  (\frac{\partial y_{ij}}{\partial f} \frac{\partial f}{\partial \beta})^2\Big)^{\frac{1}{2}}
  \\
  \frac{\partial y_{ij}}{\partial s_j} &=& f(\lambda'_{ij})
  \\
  \frac{\partial y_{ij}}{\partial f} \frac{\partial f}{\partial a_m} &=& s_j g_m(\lambda'_{ij})
  \\
  \frac{\partial y_{ij}}{\partial f} \frac{\partial f}{\partial \beta} &=& s_j a_m \frac{\partial g_m}{\partial \lambda'_{ij}}
  \frac{\partial \lambda'_{ij}}{\partial \beta} = 
  %\lambda'_{ij} &=& \lambda_i\,\sqrt{\frac{1 - \beta_j}{1 + \beta_j}}
  \quad . 
\end{eqnarray}


\paragraph{Acknowledgements:}
Thanks to DFM for helping with the model.

\end{document}
